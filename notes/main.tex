\documentclass{report}
\usepackage{amsthm, amsfonts, amsmath, amssymb}

% Custom Environments
\theoremstyle{definition}
\newtheorem{defn}{Definition}

%custom commands
% Fancy math fonts, requires amsfonts and amsmath
%% Elephant Letters
\newcommand{\A}{\mathbb{A}}
\newcommand{\B}{\mathbb{B}}
\newcommand{\Ch}{\mathbb{C}}
\newcommand{\D}{\mathbb{D}}
\newcommand{\E}{\mathbb{E}}
\newcommand{\F}{\mathbb{F}}
\newcommand{\Gh}{\mathbb{G}}
\newcommand{\Hh}{\mathbb{H}}
\newcommand{\I}{\mathbb{I}}
\newcommand{\J}{\mathbb{J}}
\newcommand{\K}{\mathbb{K}}
\newcommand{\Lh}{\mathbb{L}}
\newcommand{\M}{\mathbb{M}}
\newcommand{\N}{\mathbb{N}}
\newcommand{\Oh}{\mathbb{O}}
\newcommand{\Ph}{\mathbb{P}}
\newcommand{\Q}{\mathbb{Q}}
\newcommand{\R}{\mathbb{R}}
\newcommand{\Sh}{\mathbb{S}}
\newcommand{\T}{\mathbb{T}}
\newcommand{\Uh}{\mathbb{U}}
\newcommand{\V}{\mathbb{V}}
\newcommand{\W}{\mathbb{W}}
\newcommand{\X}{\mathbb{X}}
\newcommand{\Y}{\mathbb{Y}}
\newcommand{\Z}{\mathbb{Z}}

%% Calligrafical Letters
\newcommand{\Ac}{\mathcal{A}}
\newcommand{\BB}{\mathcal{B}}
\newcommand{\CC}{\mathcal{C}}
\newcommand{\DD}{\mathcal{D}}
\newcommand{\EE}{\mathcal{E}}
\newcommand{\FF}{\mathcal{F}}
\newcommand{\GG}{\mathcal{G}}
\newcommand{\HH}{\mathcal{H}}
\newcommand{\II}{\mathcal{I}}
\newcommand{\JJ}{\mathcal{J}}
\newcommand{\KK}{\mathcal{K}}
\newcommand{\LL}{\mathcal{L}}
\newcommand{\MM}{\mathcal{M}}
\newcommand{\NN}{\mathcal{N}}
\newcommand{\OO}{\mathcal{O}}
\newcommand{\PP}{\mathcal{P}}
\newcommand{\QQ}{\mathcal{Q}}
\newcommand{\RR}{\mathcal{R}}
\newcommand{\Sc}{\mathcal{S}}
\newcommand{\TT}{\mathcal{T}}
\newcommand{\UU}{\mathcal{U}}
\newcommand{\VV}{\mathcal{V}}
\newcommand{\WW}{\mathcal{W}}
\newcommand{\XX}{\mathcal{X}}
\newcommand{\YY}{\mathcal{Y}}
\newcommand{\ZZ}{\mathcal{Z}}


\title{Notes}
\author{Fabio Brau}

\begin{document}
\maketitle
\tableofcontents
\chapter{Markov Chains}
In the section we will introduce the model of a classical Markov chain with
discrete time for which each state belongs to some finite or countable set of
possible state. In the next section we extend the definition to states with a
continuous states or in general a continuous density function.
\section{Background}
Let $\left( \Omega, \FF, \Ph \right)$ a probability space. The class-function $\Ph$
is a finite measure over the sigma-algebra $\FF$ such that
$\Ph\left(\Omega\right)=1$. A discrete random variable is represented by a
measurable function $X:\Omega\to S$, where $S=\left\{ s_1,\dots,s_n,\dots
\right\}$ equipped with an uniform. The variable is associated to a discrete
distribution $\lambda$.
\begin{defn}[Discrete Distribution]
A sequence $\lambda=\left( \lambda_0,\dots,\lambda_n,\dots \right)$ is a
\textit{discrete distribution} if and only if $\sum_{i\in\N} \lambda_i = 1$.
\end{defn}

We will say that a random variable $X$ has distribution $\lambda$ if and only
if for each possible outcome $i\in\N$, 
\begin{equation}
  \Ph(X=i):= \Ph\left( \left\{ w\in\Omega\,:\, X(w)=i \right\}
  \right)=\lambda_i.
  \label{eq:discrete-distribution}
\end{equation}
\section{Countable States Markov Chains}
Let $\left( X_t \right)_{t\in\N}$ a sequence of random variables. Let us
assume that each instant $t\in\N$ the variable $X_t$ takes values in a countable state space $S$. 
\begin{defn}[Markov Property]
The sequence $(X_t)_{t\in\N}$ satisfies the Markov property if for each time
$t$ and for each states $s,s_0,\dots,s_n\in S$
\begin{equation}
  \Ph(X_{n+1}=s\,\vert\,
  X_0=s_0,\dots,X_n=s_n)=\Ph(X_{n+1}=s\,\vert\, X_n=s_n).
  \label{eq:markov-property}
\end{equation}
That is, the state assumed at a certain instant $t$ only depends on the
previous state and not on the whole history.
\end{defn}

Observe that, since we are assuming that $S$ is finite, then we are assuming
that there exists an enumeration $S=\left\{ s_1,\dots,s_n,\dots \right\}$.
Hence,  for the sake of simplicity, and without loss of generality, we can
assume that $S=\N$, from which $X_t\in\N$ for each $t$.

Based on the latter assumption over $S$, a \textit{Markov Chain}  with
discrete time and countable state set is represented by a tuple 
$\left( \lambda,P \right)$ as stated in the following definition.

\begin{defn}[Transaction Matrix]
  A transaction matrix $P=\left( p_{ij} \right)$ is matrix with infinite entries 
  such that
  \begin{equation}
    \forall i\in\N,\quad \sum_{j\in\N} p_{ij} = 1.
    \label{eq:transaction-matrix}
  \end{equation}
  In other words, a matrix $P$ is a transaction matrix if every row $P_{i:}$
  is a discrete distribution.
\end{defn}
\begin{defn}[Markov Chain]
  Let $\lambda$ be a discrete distribution, and let $P$ be a transaction
  matrix. A sequence of random variable $\left( x_i \right)_{i\in\N}$ is a
  Markov chain with initial distribution $\lambda$ and transaction matrix
  $P$ 
  \begin{itemize}
    \item $\lambda_i = \Ph\left( X_0=i \right)$;
    \item $p_{ij} = \Ph\left( X_n=j\,\vert\, X_{n-1}=i \right)$ for each
      state $n$.
    \item $X_{n+1}$ is independent from $X_0,\dots,X_{n-1}$.
  \end{itemize}
\end{defn}

\section{Markov Chain with Continuous State Space}

Let assume $\left(X_t \right)_{t\in\N}$ a sequence of continuous random
variables, and that the initial variable $X_t$ as a continuous
distribution $p(x)$. Remember that this implies that 
\begin{equation}
  \foreach in 
  \label{<+label+>}
\end{equation}<++>
\end{document}
