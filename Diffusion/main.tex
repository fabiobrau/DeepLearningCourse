\documentclass[aspectratio=169, 10pt]{beamer}
\usepackage{bbm}
\usepackage{algorithm2e}
\usepackage{mathtools}
\usepackage{graphicx}
\usepackage{animate}
\usepackage{hyperref}
\renewcommand\appendixname{Appendix}
\usepackage{xcolor}
\usepackage{tikz}
\usepackage{marvosym}
\usepackage{fontawesome}
\colorlet{rred}{red!80!black}
\colorlet{ggreen}{green!80!black}
\colorlet{grey}{black!50!white}

\usetheme[progressbar=foot]{metropolis}
\usepackage{appendixnumberbeamer}
\setbeamercovered{dynamic}

\usepackage{booktabs}
\usepackage[scale=2]{ccicons}
\usepackage[style=authortitle,backend=bibtex]{biblatex}
\addbibresource{biblio.bib}

\usepackage{pgfplots}
\usepgfplotslibrary{dateplot}
\setbeamertemplate{caption}{\raggedright\insertcaption\par}
\setlength{\abovecaptionskip}{-3pt plus 0pt minus 0pt}

\usepackage{xspace}
\theoremstyle{definition}
\newtheorem{defn}{Definition}
\newtheorem{obs}{Observation}

% math symbols
% Fancy math fonts, requires amsfonts and amsmath
%% Elephant Letters
\newcommand{\A}{\mathbb{A}}
\newcommand{\B}{\mathbb{B}}
\newcommand{\Ch}{\mathbb{C}}
\newcommand{\D}{\mathbb{D}}
\newcommand{\E}{\mathbb{E}}
\newcommand{\F}{\mathbb{F}}
\newcommand{\Gh}{\mathbb{G}}
\newcommand{\Hh}{\mathbb{H}}
\newcommand{\I}{\mathbb{I}}
\newcommand{\J}{\mathbb{J}}
\newcommand{\K}{\mathbb{K}}
\newcommand{\Lh}{\mathbb{L}}
\newcommand{\M}{\mathbb{M}}
\newcommand{\N}{\mathbb{N}}
\newcommand{\Oh}{\mathbb{O}}
\newcommand{\Ph}{\mathbb{P}}
\newcommand{\Q}{\mathbb{Q}}
\newcommand{\R}{\mathbb{R}}
\newcommand{\Sh}{\mathbb{S}}
\newcommand{\T}{\mathbb{T}}
\newcommand{\Uh}{\mathbb{U}}
\newcommand{\V}{\mathbb{V}}
\newcommand{\W}{\mathbb{W}}
\newcommand{\X}{\mathbb{X}}
\newcommand{\Y}{\mathbb{Y}}
\newcommand{\Z}{\mathbb{Z}}

%% Calligrafical Letters
\newcommand{\Ac}{\mathcal{A}}
\newcommand{\BB}{\mathcal{B}}
\newcommand{\CC}{\mathcal{C}}
\newcommand{\DD}{\mathcal{D}}
\newcommand{\EE}{\mathcal{E}}
\newcommand{\FF}{\mathcal{F}}
\newcommand{\GG}{\mathcal{G}}
\newcommand{\HH}{\mathcal{H}}
\newcommand{\II}{\mathcal{I}}
\newcommand{\JJ}{\mathcal{J}}
\newcommand{\KK}{\mathcal{K}}
\newcommand{\LL}{\mathcal{L}}
\newcommand{\MM}{\mathcal{M}}
\newcommand{\NN}{\mathcal{N}}
\newcommand{\OO}{\mathcal{O}}
\newcommand{\PP}{\mathcal{P}}
\newcommand{\QQ}{\mathcal{Q}}
\newcommand{\RR}{\mathcal{R}}
\newcommand{\Sc}{\mathcal{S}}
\newcommand{\TT}{\mathcal{T}}
\newcommand{\UU}{\mathcal{U}}
\newcommand{\VV}{\mathcal{V}}
\newcommand{\WW}{\mathcal{W}}
\newcommand{\XX}{\mathcal{X}}
\newcommand{\YY}{\mathcal{Y}}
\newcommand{\ZZ}{\mathcal{Z}}


\DeclareMathOperator*{\argmin}{argmin}
\DeclareMathOperator*{\argmax}{argmax}


\title{Diffusion Models: DALL-E\\ 
  \large{Deep Learning and Neural Networks: Advanced Topics}}
\date{March 1, 2023}
\author{Fabio Brau}
\institute{Scuola Superiore Sant'Anna, Pisa.}
% \titlegraphic{\hfill\includegraphics[height=1.5cm]{logo.pdf}}

\setbeamertemplate{background}{%
    \begin{picture}(300,253)
      \hspace{14.45cm}
       \includegraphics[scale=0.1]{pic/logoretis_noname.png}
   \end{picture}
}
\setbeamercolor{background canvas}{bg=white}
\begin{document}
{\setbeamertemplate{background}{%
    \begin{picture}(300,240)
      \hspace{0.9cm}
       \includegraphics[scale=0.35]{pic/tecip_logo-ENG.png}
       \hspace{0.5cm}
       \includegraphics[scale=0.085]{pic/logoretis.png}
   \end{picture}}%
\maketitle
}
\begin{frame}
  \tableofcontents
\end{frame}
\section{Introduction}
\section{Diffusion Models}
\begin{frame}{Overview}
  \begin{center}
    \it
    Diffusion models are generative models that aim at denoising data
  \end{center}
  \begin{figure}[h!]
    \centering
    \includegraphics[scale=0.2]{./pic/diffusion_intro.png}
  \end{figure}
\end{frame}
\begin{frame}{Timeline}
\begin{enumerate}
  \item[\bf 2015)] \textit{\ldots Non-equilibrium Thermodynamics}. Sohl-Dickstein
    et al. ICML\vfill
  \item[\bf 2020)] \textit{Denoising Diffusion Probabilistic Models}.
  Ho et al. NeurIPS.\vfill
  \item[\bf 2021)] \textit{Score-Based Generative Modeling Through SDE}. Song et
    al. ICLR.
\end{enumerate}
\end{frame}
\begin{frame}{Deep Unsupervised Learning using Non-Equilibrium Thermodynamics}
  \begin{figure}[h!]
    \centering
    \includegraphics[scale=.23]{./pic/thermodynimc.png}
    \caption{Diffusion process as a \textbf{Markov Chain} with \textbf{Continuous
    State Space} and \textbf{Discrete Time}.\footcite{thermodynamic}}
  \end{figure}
\end{frame}
\begin{frame}{Reminder: Markov Chains with Discrete Time}
\textbf{Informal Definition}\\ 
A sequence of random variables $\xx^{(0)},\xx^{(1)},\cdots,\xx^{(t)},\cdots$,
such that:
\begin{itemize}
  \item $\xx^{(t)}\in S$, where $S$ \textbf{State Space}
  \item The future $\xx^{(t+1)}$ depends on the present $\xx^{(t)}$ 
    but not on the past $\xx^{(t-1)}$
\end{itemize}
  \vfill
  \begin{minipage}[t]{0.4\textwidth}
    \begin{center}
      \textbf{Discrete State Space $S$}
    \end{center}
  \end{minipage}\hfill%
  \begin{minipage}[t]{0.4\textwidth}
    \begin{center}
      \textbf{Continuous State Space $S$}
    \end{center}
  \end{minipage}
\end{frame}

\begin{frame}{Reminder: MCDT with Discrete State Space}
  \begin{minipage}[t]{0.45\textwidth}
    \textbf{Definition}\\
    A sequence of random variables $\left\{ \xx^{(t)}\right\}_{t\in\TT}
    \subseteq S$
    \begin{itemize}
      \item \textbf{Discrete Time Property}\\
      \(
        \xx^{(0)},\,\xx^{(1)},\cdots,\xx^{(t)},\cdots 
      \)
      \item \textbf{Markov Property}\\
      \(
        \Ph\left( \xx^{(t+1)}\in A\,\vert\, \xx^{(0)},\ldots,\xx^{(t)}
        \right)=\Ph\left(\xx^{(t+1)}\in A\,\vert\,\xx^{(t)} \right)
      \)
    \end{itemize}
  \end{minipage}\hfill%
  \begin{minipage}[t]{0.5\textwidth}
    \begin{figure}[h!]
      \centering
      \includegraphics[scale=0.1]{./pic/mcdt_discrete_space.png}
      \caption{Two MCs with a discrete state space of respectively $2$ and
      $3$ states each.}
    \end{figure}
  \end{minipage}
\end{frame}
\begin{frame}{Reminder: DTMC with Continuous State Space}
  Let assume $\xx,\yy\in S$ where $S$ continuous state space (e.g.
  $S=\R^d$)\.\
  \textbf{Joint Distribution} $p(\xx,\yy)$
  \[
    \Ph\left( \xx\in A\,\vert\,\yy\in B \right)&=
    \int_A\int_B p\left( \xx,\yy \right)\,d\xx\,d\yy
  \]

  \textbf{Transactional Kernel} $p(\xx\,\vert\,\yy)$
  \[
    p\left( \xx,\yy \right) = p(\xx\,\vert\,\yy)\,p\left( \xx \right)
  \]
  \textbf{Marginal Distribution} $p(\xx)$
  \[
    p\left( \xx \right)=\int_S p\left( \xx,\yy \right)\,d\yy = \int_S p(\xx\,\vert\,\yy)\,p\left( \xx \right)\,d\yy 
  \]

\end{frame}

\begin{frame}{Markov Chains with Discrete Time}
\textbf{Definition}\\ 
  A sequence of random variables $\left\{ \xx^{(t)}\right\}_{t\in\TT}
  \subseteq S$, such that the future $\xx^{(t+1)}$ depends on
  the present $\xx^{(t)}$ but not on the past $\xx^{(t-1)}$.\\
  \begin{itemize}
    \item \textbf{Discrete Time Property}\\
      \(
        \xx^{(0)},\,\xx^{(1)},\cdots,\xx^{(t)},\cdots 
      \)
    \item \textbf{Markov Property}\\
      \(
        \Ph\left( \xx^{(t+1)}\in A\,\vert\, \xx^{(0)},\ldots,\xx^{(t)}
        \right)=\Ph\left(\xx^{(t+1)}\in A\,\vert\,\xx^{(t)} \right)
      \)
  \end{itemize}
  \vfill
  \begin{minipage}[t]{0.4\textwidth}
    \begin{center}
      \textbf{Discrete State Space $S$}
    \end{center}
  \end{minipage}\hfill%
  \begin{minipage}[t]{0.4\textwidth}
    \begin{center}
      \textbf{Continuous State Space $S$}
    \end{center}
  \end{minipage}
\end{frame}
\section{Broader Impacts}
\begin{frame}{CLIP Model}
  \begin{center}
    \it
    ``We also found discrepancies across gender and race for people
    categorized into the ‘crime’ and ‘non-human’
    categories\ldots''\footcite{clip}
  \end{center}
\end{frame}
{%
  \setbeamercolor{background canvas}{bg=black!10!white}
  \setbeamertemplate{background}{%
  \begin{picture}(300,240)
    \hspace{0.9cm}
    \includegraphics[scale=0.35]{pic/tecip_logo-ENG.png}
    \hspace{0.5cm}
    \includegraphics[scale=0.17]{pic/dipe.png}
    \hspace{0.5cm}
    \includegraphics[scale=0.085]{pic/logoretis.png}
  \end{picture}}
\begin{frame}{}
  \textbf{\Huge Thanks for the attention}\\
  \vspace{20pt}
  \begin{minipage}[h]{0.6\textwidth}
  {\large\bf Fabio Brau}
  \vspace{5pt}
  \begin{itemize}
    \item[\faUniversity] {\bf Scuola Superiore Sant'Anna, Pisa}
    \item[\Letter] \texttt{fabio.brau@santannapisa.it}
    \item[\faGlobe] \href{http://retis.santannapisa.it/~f.brau/}{\tt %
                    retis.santannapisa.it/\textasciitilde f.brau}
    \item[\faLinkedin]
      \href{https://www.linkedin.com/in/fabio-brau}{\tt%
        linkedin.com/in/fabio-brau}
  \end{itemize}
  \end{minipage}
\end{frame}
}
\end{document}
